\documentclass[fontsize=9, a5paper]{scrbook}
\usepackage{fancyhdr}
\usepackage[a5paper]{geometry}
\usepackage{ifthen}
\usepackage{keyval}
\usepackage{multicol}
\usepackage{makeidx}
\usepackage{mparhack}
\usepackage{poemscol}
\usepackage{afterpage}

\newcommand\blankpage{%
    \null
    \thispagestyle{empty}%
    \addtocounter{page}{-1}%
    \newpage}
\fancyhf{}
\cfoot{\thepage}
\pagestyle{plain}
\stanzaatbottom{*}
\indexingontrue
\setverselinemodulo{5}
\begin{document}

\begin{maintitlepage}
\wholebooktitle{\textbf{I.} A Volume of Stitched Lays}
\noindent\fbox{
	\parbox{\textwidth}{
		\poemtitle{The Human Seasons}
		\attribution{John Keats (31 Oct 1795 - 23 Feb 1821)}
		\index{The Human Seasons}
		\begin{poem}
		    \begin{stanza}
		        \textsc{Four Seasons} fill the measure of the year;\verseline
			\index{\textsc{Four Seasons} fill the measure of the year;}
		        \verseindent There are four seasons in the mind of man:\verseline
		        He has his lusty Spring, when fancy clear\verseline
		             \verseindent Takes in all beauty with an easy span: \verseline
		        He has his Summer, when luxuriously \verseline
		             \verseindent Spring's honied cud of youthful thought he loves \verseline
		        To ruminate, and by such dreaming high \verseline
		             \verseindent Is nearest unto heaven: quiet coves \verseline
		        His soul has in its Autumn, when his wings \verseline
		             \verseindent He furleth close; contented so to look \verseline
		        On mists in idleness—to let fair things \verseline
		             \verseindent Pass by unheeded as a threshold brook. \verseline
		        He has his Winter too of pale misfeature, \verseline
		        Or else he would forego his mortal nature.
 		   \end{stanza}
		\end{poem}
	}
}
\end{maintitlepage}

\pmclcontentsname={Rhapsodies}
\putpoemcontents
\makepoemcontents

\pagenumbering{arabic}
\afterpage{\blankpage}
\afterpage{\blankpage}
\begin{volumetitlepage}
\sequencetitle{\textbf{I.} Spring}
\bigskip
\noindent\fbox{
	\parbox{\textwidth}{
		\poemtitle{\textbf{A Poet! He Hath Put His Heart to School!}}
		\attribution{William Wordsworth}
			\begin{poem}
				\begin{stanza}
					 \textsc{A poet!}—He hath put his heart to school,\verseline
					Nor dares to move unpropped upon the staff\verseline
					Which art hath lodged within his hand—must laugh\verseline
					By precept only, and shed tears by rule.\verseline
					Thy Art be Nature; the live current quaff,\verseline
					And let the groveller sip his stagnant pool,\verseline
					In fear that else, when Critics grave and cool\verseline
					Have killed him, Scorn should write his epitaph.\verseline
					How does the Meadow-flower its bloom unfold?\verseline
					Because the lovely little flower is free\verseline
					Down to its root, and, in that freedom, bold;\verseline
					And so the grandeur of the Forest-tree\verseline
					Comes not by casting in a formal mould,\verseline
					But from its own divine vitality.
				\end{stanza}
			\end{poem}
	}
}

\end{volumetitlepage}

\pagebreak

\poemtitle{(Spring 1) \textbf{if everything happens that can't be done}}
\attribution{e e cummings (14 Oct 1894 - 3 Sep 1962)}
\begin{poem}
	\begin{stanza}
		\textsc{if everything} happens that can't be done\verseline
		(and anything's righter\verseline
		than books\verseline
		could plan)\verseline
		the stupidest teacher will almost guess\verseline
		(with a run\verseline
		skip\verseline
		around we go yes)\verseline
		there's nothing as something as one
	\end{stanza}

	\begin{stanza}
		\textsc{one hasn't} a why or because or although\verseline
		(and buds know better\verseline
		than books\verseline
		don't grow)\verseline
		one's anything old being everything new\verseline
		(with a what\verseline
		which\verseline
		around we come who)\verseline
		one's everyanything so
	\end{stanza}

	\begin{stanza}
		\textsc{so world} is a leaf so a tree is a bough\verseline
		(and birds sing sweeter\verseline
		than books\verseline
		tell how)\verseline
		so here is away and so your is a my\verseline
		(with a down\verseline
		up\verseline
		around again fly)\verseline
		forever was never till now
	\end{stanza}

	\pagebreak

	\begin{stanza}
		\textsc{now i} love you and you love me\verseline
		(and books are shuter\verseline
		than books\verseline
		can be)\verseline
		and deep in the high that does nothing but fall\verseline
		(with a shout\verseline
		each\verseline
		around we go all)\verseline
		there's somebody calling who's we
	\end{stanza}

	\begin{stanza}
		\textsc{we're anything} brighter than even the sun\verseline
		(we're everything greater\verseline
		than books\verseline
		might mean)\verseline
		we're everyanything more than believe\verseline
		(with a spin\verseline
		leap\verseline
		alive we're alive)\verseline
		we're wonderful one times one
	\end{stanza}
\end{poem}

\pagebreak

\poemtitle{(Spring 2) \textbf{The Lover: A Ballad}}
\attribution{Lady Mary Wortley Montagu (15 May 1689 - 21 Aug 1762)}
\begin{poem}
	\begin{stanza}
		\textsc{At length,} by so much importunity press'd,\verseline
		Take, C——, at once, the inside of my breast;\verseline
		This stupid indiff'rence so often you blame,\verseline
		Is not owing to nature, to fear, or to shame:\verseline
		I am not as cold as a virgin in lead,\verseline
		Nor is Sunday's sermon so strong in my head:\verseline
		I know but too well how time flies along,\verseline
		That we live but few years, and yet fewer are young.
	\end{stanza}
	
	\begin{stanza}
		\textsc{But I} hate to be cheated, and never will buy\verseline
		Long years of repentance for moments of joy,\verseline
		Oh! was there a man (but where shall I find\verseline
		Good sense and good nature so equally join'd?)\verseline
		Would value his pleasure, contribute to mine;\verseline
		Not meanly would boast, nor would lewdly design;\verseline
		Not over severe, yet not stupidly vain,\verseline
		For I would have the power, tho' not give the pain.
	\end{stanza}
	
	\begin{stanza}
		\textsc{No pedant,} yet learned; no rake-helly gay,\verseline
		Or laughing, because he has nothing to say;\verseline
		To all my whole sex obliging and free,\verseline
		Yet never be fond of any but me;\verseline
		In public preserve the decorum that's just,\verseline
		And shew in his eyes he is true to his trust;\verseline
		Then rarely approach, and respectfully bow,\verseline
		But not fulsomely pert, nor yet foppishly low.
	\end{stanza}

	\pagebreak
	
	\begin{stanza}
		\textsc{But when} the long hours of public are past,\verseline
		And we meet with champagne and a chicken at last,\verseline
		May ev'ry fond pleasure that moment endear;\verseline
		Be banish'd afar both discretion and fear!\verseline
		Forgetting or scorning the airs of the crowd,\verseline
		He may cease to be formal, and I to be proud.\verseline
		Till lost in the joy, we confess that we live,\verseline
		And he may be rude, and yet I may forgive.
	\end{stanza}
	
	\begin{stanza}
		\textsc{And that} my delight may be solidly fix'd,\verseline
		Let the friend and the lover be handsomely mix'd;\verseline
		In whose tender bosom my soul may confide,\verseline\
		Whose kindness can soothe me, whose counsel can guide.\verseline
		From such a dear lover as here I describe,\verseline
		No danger should fright me, no millions should bribe;\verseline
		But till this astonishing creature I know,\verseline
		As I long have liv'd chaste, I will keep myself so.
	\end{stanza}
	
	\begin{stanza}
		\textsc{I never} will share with the wanton coquette,\verseline
		Or be caught by a vain affectation of wit.\verseline
		The toasters and songsters may try all their art,\verseline
		But never shall enter the pass of my heart.\verseline
		I loath the lewd rake, the dress'd fopling despise:\verseline
		Before such pursuers the nice virgin flies:\verseline
		And as Ovid has sweetly in parable told,\verseline
		We harden like trees, and like rivers grow cold. 
	\end{stanza}
\end{poem}

\pagebreak

\afterpage{\blankpage}
\poemtitle{(Spring 3) \textbf{The Sun Rising}}
\attribution{John Donne (22 Jan 1572 - 31 Mar 1631)}
\begin{poem}
	\begin{stanza}
		               \verseindent\verseindent \textsc{Busy old} fool, unruly sun,\verseline
		               \verseindent\verseindent Why dost thou thus,\verseline
		Through windows, and through curtains call on us?\verseline
		Must to thy motions lovers' seasons run?\verseline
		               \verseindent\verseindent Saucy pedantic wretch, go chide\verseline
		               \verseindent\verseindent Late school boys and sour prentices,\verseline
		         \verseindent Go tell court huntsmen that the king will ride,\verseline
		         \verseindent Call country ants to harvest offices,\verseline
		Love, all alike, no season knows nor clime,\verseline
		Nor hours, days, months, which are the rags of time.
	\end{stanza}
	
	\begin{stanza}
		               \verseindent\verseindent \textsc{Thy beams,} so reverend and strong\verseline
		               \verseindent\verseindent Why shouldst thou think?\verseline
		I could eclipse and cloud them with a wink,\verseline\
		But that I would not lose her sight so long;\verseline
		               \verseindent\verseindent If her eyes have not blinded thine,\verseline
		               \verseindent\verseindent Look, and tomorrow late, tell me,\verseline
		         \verseindent Whether both th' Indias of spice and mine\verseline
		         \verseindent Be where thou leftst them, or lie here with me.\verseline
		Ask for those kings whom thou saw'st yesterday,\verseline
		And thou shalt hear, All here in one bed lay.
	\end{stanza}
		
	\begin{stanza}
		               \verseindent\verseindent \textsc{She's all} states, and all princes, I,\verseline
		               \verseindent\verseindent Nothing else is.\verseline
		Princes do but play us; compared to this,\verseline
		All honor's mimic, all wealth alchemy.\verseline
		               \verseindent\verseindent Thou, sun, art half as happy as we,\verseline
		               \verseindent\verseindent In that the world's contracted thus.\verseline
		         \verseindent Thine age asks ease, and since thy duties be\verseline
		         \verseindent To warm the world, that's done in warming us.\verseline
		Shine here to us, and thou art everywhere;\verseline
		This bed thy center is, these walls, thy sphere.
	\end{stanza}
\end{poem}

\pagebreak

\afterpage{\blankpage}
\poemtitle{(Spring 4) \textbf{Heart and Service}}
\attribution{Sir Thomas Wyatt (?? ?? 1503 - 11 Oct 1542)}
\begin{poem}
	\begin{stanza}
		\textsc{The heart} and service to you proffer'd\verseline
		With right good will full honestly,\verseline
		Refuse it not, since it is offer'd,\verseline
		But take it to you gentlely.
	\end{stanza}
	
	\begin{stanza}
		\textsc{And though} it be a small present,\verseline
		Yet good, consider graciously\verseline
		The thought, the mind, and the intent\verseline
		Of him that loves you faithfully.
	\end{stanza}
	
	\begin{stanza}
		\textsc{It were} a thing of small effect\verseline
		To work my woe thus cruelly,\verseline
		For my good will to be abject:\verseline
		Therefore accept it lovingly.
	\end{stanza}
	
	\begin{stanza}
		\textsc{Pain or} travel, to run or ride,\verseline
		I undertake it pleasantly;\verseline
		Bid ye me go, and straight I glide\verseline
		At your commandement humbly.
	\end{stanza}
	
	\begin{stanza}
		\textsc{Pain or} pleasure, now may you plant\verseline
		Even which it please you steadfastly;\verseline
		Do which you list, I shall not want\verseline
		To be your servant secretly.
	\end{stanza}
	
	\begin{stanza}
		\textsc{And since} so much I do desire\verseline
		To be your own assuredly,\verseline
		For all my service and my hire\verseline
		Reward your servant liberally.
	\end{stanza}
\end{poem}

\pagebreak


\poemtitle{(Spring 5) \textbf{Desiderata}}
\attribution{Max Ehrmann (26 Sep 1872 - 9 Sep 1945)}
\begin{poem}
	\begin{stanza}
		\textsc{Go placidly} amid the noise and the haste, \verseline
			\verseindent and remember what peace there may be \verseline
			\verseindent\verseindent in silence.\verseline
		 As far as possible, without surrender, \verseline
		\verseindent be on good terms with all persons.
	\end{stanza}
	
	\begin{stanza}
		\textsc{Speak your} truth quietly and clearly; \verseline
			\verseindent and listen to others, \verseline
			\verseindent even to the dull and the ignorant; \verseline
			\verseindent they too have their story.
	\end{stanza}
	
	\begin{stanza}
		\textsc{Avoid loud} and aggressive persons; \verseline
			\verseindent they are vexatious to the spirit. \verseline
		If you compare yourself with others, \verseline
			\verseindent you may become vain or bitter, \verseline
		\verseindent for always there will be \verseline
			\verseindent \verseindent greater and lesser \verseline
				\verseindent \verseindent\verseindent persons than yourself. 
	\end{stanza}
	
	\begin{stanza}
		\textsc{Enjoy your} achievements as well as your plans. \verseline
		Keep interested in your own career, however humble; \verseline
		it is a real possession in the changing fortunes of time.
	\end{stanza}
	
	\begin{stanza}
		\textsc{Exercise caution} in your business affairs, \verseline
			\verseindent for the world is full of trickery. \verseline
		But let this not blind you to what virtue there is; \verseline
			\verseindent many persons strive for high ideals, \verseline
			\verseindent and everywhere life is full of heroism.
	\end{stanza}
	
	\begin{stanza}
		\textsc{Be yourself.} \verseline
		Especially do not feign affection. \verseline
		Neither be cynical about love; \verseline
			\verseindent for in the face of all aridity \verseline
			\verseindent and disenchantment, \verseline
			\verseindent it is as perennial as the grass.
	\end{stanza}
	
	\begin{stanza}
		\textsc{Take kindly} the counsel of the years, \verseline
			\verseindent gracefully surrendering the things of youth.
	\end{stanza}
	
	\begin{stanza}
		\textsc{Nurture strength} of spirit to shield you \verseline
			\verseindent in sudden misfortune. \verseline
		But do not distress yourself \verseline
			\verseindent with dark imaginings. \verseline
		Many fears are born of fatigue \verseline
			\verseindent and loneliness.
	\end{stanza}
	
	\begin{stanza}
		\textsc{Beyond a} wholesome discipline, be gentle with yourself. \verseline
		You are a child of the universe \verseline
			\verseindent no less than the trees and the stars; \verseline
			\verseindent \verseindent you have a right to be here.
	\end{stanza}

	\begin{stanza}
		\textsc{And whether} or not it is clear to you, \verseline
			\verseindent no doubt the universe is unfolding as it should. \verseline
		Therefore be at peace with God, \verseline
			\verseindent whatever you conceive Him to be. \verseline
		And whatever your labors and aspirations, \verseline
			\verseindent in the noisy confusion of life, \verseline
				\verseindent\verseindent keep peace in your soul. \verseline
		With all its sham, drudgery and broken dreams, \verseline
			\verseindent it is still a beautiful world. \verseline
		Be cheerful. 
	\end{stanza}

	\begin{stanza}
		\textsc{Strive to be happy.}
	\end{stanza}
\end{poem}

\pagebreak

\poemtitle{(Spring 6) \textbf{\textit{I Met a Fool}}}
\attribution{Jaques, \underline{As You Like It} by William Shakespeare (?? Apr 1564 - 23 Apr 1616)}
\begin{poem}
	\begin{stanza}
		\textsc{A fool}, a fool! I met a fool i' the forest,\verseline
		A motley fool; a miserable world!\verseline
		As I do live by food, I met a fool\verseline
		Who laid him down and bask'd him in the sun,\verseline
		And rail'd on Lady Fortune in good terms,\verseline
		In good set terms and yet a motley fool.\verseline
		'Good morrow, fool,' quoth I. 'No, sir,' quoth he,\verseline
		'Call me not fool till heaven hath sent me fortune:'\verseline
		And then he drew a dial from his poke,\verseline
		And, looking on it with lack-lustre eye,\verseline
		Says very wisely, 'It is ten o'clock:\verseline
		Thus we may see,' quoth he, 'how the world wags:\verseline
		'Tis but an hour ago since it was nine,\verseline
		And after one hour more 'twill be eleven;\verseline
		And so, from hour to hour, we ripe and ripe,\verseline
		And then, from hour to hour, we rot and rot;\verseline
		And thereby hangs a tale.' When I did hear\verseline
		The motley fool thus moral on the time,\verseline
		My lungs began to crow like chanticleer,\verseline
		That fools should be so deep-contemplative,\verseline
		And I did laugh sans intermission\verseline
		An hour by his dial. O noble fool!\verseline
		A worthy fool! Motley's the only wear.
	\end{stanza}

	\pagebreak	

	\begin{stanza}
		\textsc{O worthy} fool! One that hath been a courtier,\verseline
		And says, if ladies be but young and fair,\verseline
		They have the gift to know it: and in his brain,\verseline
		Which is as dry as the remainder biscuit\verseline
		After a voyage, he hath strange places cramm'd\verseline
		With observation, the which he vents\verseline
		In mangled forms. O that I were a fool!\verseline
		I am ambitious for a motley coat.
	\end{stanza}
	
	\begin{stanza}
		\textsc{It is} my only suit;\verseline
		Provided that you weed your better judgments\verseline
		Of all opinion that grows rank in them\verseline
		That I am wise. I must have liberty\verseline
		Withal, as large a charter as the wind,\verseline
		To blow on whom I please; for so fools have;\verseline
		And they that are most galled with my folly,\verseline
		They most must laugh. And why, sir, must they so?\verseline
		The 'why' is plain as way to parish church:\verseline
		He that a fool doth very wisely hit\verseline
		Doth very foolishly, although he smart,\verseline
		Not to seem senseless of the bob: if not,\verseline
		The wise man's folly is anatomized\verseline
		Even by the squandering glances of the fool.\verseline
		Invest me in my motley; give me leave\verseline
		To speak my mind, and I will through and through\verseline
		Cleanse the foul body of the infected world,\verseline
		If they will patiently receive my medicine.
	\end{stanza}
\end{poem}

\pagebreak

\begin{volumetitlepage}
\sequencetitle{\textbf{II.} Summer}
\bigskip
\noindent\fbox{
	\parbox{\textwidth}{
		\poemtitle{\textbf{A Drinking Song}}
		\attribution{William Butler Yeats}
		\begin{poem}
			\begin{stanza}
				\textsc{Wine comes} in at the mouth\verseline
				And love comes in at the eye;\verseline
				That’s all we shall know for truth\verseline
				Before we grow old and die.\verseline
				I lift the glass to my mouth,\verseline
				I look at you, and I sigh.
			\end{stanza}
		\end{poem}
	}
}
\end{volumetitlepage}

\pagebreak

\afterpage{\blankpage}
\poemtitle{(Summer 1) \textbf{\textit{All Delights Are Vain}}}
\attribution{Berowne, \underline{Love's Labors Lost} by William Shakespeare (?? Apr 1564 - 23 Apr 1616)}
\begin{poem}
	\begin{stanza}
		\textsc{Why, all} delights are vain, and that most vain\verseline
		Which with pain purchased doth inherit pain:\verseline
		As painfully to pore upon a book\verseline
		 \verseindent To seek the light of truth, while truth the while\verseline
		Doth falsely blind the eyesight of his look.\verseline
		 \verseindent Light seeking light doth light of light beguile.\verseline
		So, ere you find where light in darkness lies,\verseline
		Your light grows dark by losing of your eyes.\verseline
		Study me how to please the eye indeed\verseline
		 \verseindent By fixing it upon a fairer eye,\verseline
		Who dazzling so, that eye shall be his heed\verseline
		 \verseindent And give him light that it was blinded by.\verseline
		Study is like the heaven’s glorious sun,\verseline
		 \verseindent That will not be deep-searched with saucy looks.\verseline
		Small have continual plodders ever won,\verseline
		 \verseindent Save base authority from others’ books.\verseline
		These earthly godfathers of heaven’s lights,\verseline
		 \verseindent That give a name to every fixèd star,\verseline
		Have no more profit of their shining nights\verseline
		 \verseindent Than those that walk and wot not what they are.\verseline
		Too much to know is to know naught but fame,\verseline
		And every godfather can give a name.
	\end{stanza}
\end{poem}

\pagebreak

\afterpage{\blankpage}
\poemtitle{(Summer 2) \textbf{When I Heard the Learn'd Astronomer}}
\attribution{Walt Whitman (31 May 1819 - 26 Mar 1892)}
\begin{poem}
	\begin{stanza}
		\textsc{When I} heard the learn’d astronomer,\verseline
		When the proofs, \verseline
			\verseindent the figures, \verseline
				\verseindent\verseindent were ranged in columns before me,\verseline
		When I was shown the charts and diagrams, \verseline
			\verseindent to add, divide,\verseline
				\verseindent\verseindent and measure them,\verseline
		When I sitting heard the astronomer \verseline
			\verseindent where he lectured \verseline
				\verseindent\verseindent with much applause in the lecture-room,\verseline
		How soon unaccountable I became \verseline
			\verseindent tired and sick,\verseline
		Till rising and gliding out \verseline
			\verseindent I wander’d off by myself,\verseline
		In the mystical moist night-air, \verseline
			\verseindent and from time to time,\verseline
		Look’d up in perfect silence at the stars.
	\end{stanza}
\end{poem}

\pagebreak

\afterpage{\blankpage}
\poemtitle{(Summer 3) \textbf{Love's Growth}}
\attribution{John Donne (22 Jan 1572 - 31 Mar 1631)}
\begin{poem}
	\begin{stanza}
		\textsc{I scarce} believe my love to be so pure\verseline
		   \verseindent As I had thought it was,\verseline
		   \verseindent Because it doth endure\verseline
		Vicissitude, and season, as the grass;\verseline
		Methinks I lied all winter, when I swore\verseline
		My love was infinite, if spring make’ it more.
	\end{stanza}
	
	\begin{stanza}
		\textsc{But if} medicine, love, which cures all sorrow\verseline
		With more, not only be no quintessence,\verseline
		But mixed of all stuffs paining soul or sense,\verseline
		And of the sun his working vigor borrow,\verseline
		Love’s not so pure, and abstract, as they use\verseline
		To say, which have no mistress but their muse,\verseline
		But as all else, being elemented too,\verseline
		Love sometimes would contemplate, sometimes do.
	\end{stanza}
	
	\begin{stanza}
		\textsc{And yet} no greater, but more eminent,\verseline
		   \verseindent Love by the spring is grown;\verseline
		   \verseindent As, in the firmament,\verseline
		Stars by the sun are not enlarged, but shown,\verseline
		Gentle love deeds, as blossoms on a bough,\verseline
		From love’s awakened root do bud out now.
	\end{stanza}
	
	\begin{stanza}
		\textsc{If, as} water stirred more circles be\verseline
		Produced by one, love such additions take,\verseline
		Those, like so many spheres, but one heaven make,\verseline
		For they are all concentric unto thee;\verseline
		And though each spring do add to love new heat,\verseline
		As princes do in time of action get\verseline
		New taxes, and remit them not in peace,\verseline
		No winter shall abate the spring’s increase.
	\end{stanza}
\end{poem}

\pagebreak

\afterpage{\blankpage}
\poemtitle{(Summer 4) \textbf{The Time I've Lost In Wooing}}
\attribution{Thomas Moore (28 May 1779 - 25 Feb 1852)}
\begin{poem}
	\begin{stanza}
		\textsc{The time} I’ve lost in wooing,\verseline
		In watching and pursuing\verseline
		The light, that lies\verseline
		In woman’s eyes,\verseline
		Has been my heart’s undoing.\verseline
		Though Wisdom oft has sought me,\verseline
		I scorn’d the lore she brought me,\verseline
		My only books\verseline
		Were woman’s looks,\verseline
		And folly’s all they’ve taught me.
	\end{stanza}
	
	\begin{stanza}
		\textsc{Her smile} when Beauty granted,\verseline
		I hung with gaze enchanted,\verseline
		Like him the Sprite,\verseline
		Whom maids by night\verseline
		Oft meet in glen that’s haunted.\verseline
		Like him, too, Beauty won me,\verseline
		But while her eyes were on me,\verseline
		If once their ray\verseline
		Was turn’d away,\verseline
		Oh! winds could not outrun me.
	\end{stanza}
	
	\begin{stanza}
		\textsc{And are} those follies going?\verseline
		And is my proud heart growing\verseline
		Too cold or wise\verseline
		For brilliant eyes\verseline
		Again to set it glowing?\verseline
		No, vain, alas! th’ endeavour\verseline
		From bonds so sweet to sever;\verseline
		Poor Wisdom’s chance\verseline
		Against a glance\verseline
		Is now as weak as ever.
	\end{stanza}
\end{poem}

\pagebreak

\afterpage{\blankpage}
\poemtitle{(Summer 5) \textbf{Forget Not Yet The Tried Intent}}
\attribution{Sir Thomas Wyatt (?? ?? 1503 - 11 Oct 1542)}
\begin{poem}
	\begin{stanza}
		\textsc{Forget not} yet the tried intent\verseline
		Of such a truth as I have meant;\verseline
		My great travail so gladly spent,\verseline
		            \verseindent\verseindent Forget not yet.
	\end{stanza}
	
	\begin{stanza}
		  \verseindent \textsc{Forget not} yet when first began\verseline
		The weary life ye know, since whan\verseline
		The suit, the service, none tell can;\verseline
		           \verseindent\verseindent Forget not yet.
	\end{stanza}
	
	\begin{stanza}
		  \verseindent \textsc{Forget not} yet the great assays,\verseline
		The cruel wrong, the scornful ways;\verseline
		The painful patience in denays,\verseline
		           \verseindent\verseindent Forget not yet.
	\end{stanza}
	
	\begin{stanza}
		  \verseindent \textsc{Forget not} yet, forget not this,\verseline
		How long ago hath been and is\verseline
		The mind that never meant amiss;\verseline
		           \verseindent\verseindent Forget not yet.
	\end{stanza}
	
	\begin{stanza}
		  \verseindent \textsc{Forget not} then thine own approved,\verseline
		The which so long hath thee so loved,\verseline
		Whose steadfast faith yet never moved;\verseline
		          \verseindent\verseindent  Forget not this.
	\end{stanza}
\end{poem}

\pagebreak

\afterpage{\blankpage}
\poemtitle{(Summer 6) \textbf{I Wandered Lonely as a Cloud}}
\attribution{William Wordsworth (7 Apr 1770 - 23 Apr 1850)}
\begin{poem}
	\begin{stanza}
		\textsc{I wandered} lonely as a cloud\verseline
		That floats on high o'er vales and hills,\verseline
		When all at once I saw a crowd,\verseline
		A host, of golden daffodils;\verseline
		Beside the lake, beneath the trees,\verseline
		Fluttering and dancing in the breeze.
	\end{stanza}
	
	\begin{stanza}
		\textsc{Continuous as} the stars that shine\verseline
		And twinkle on the milky way,\verseline
		They stretched in never-ending line\verseline
		Along the margin of a bay:\verseline
		Ten thousand saw I at a glance,\verseline
		Tossing their heads in sprightly dance.
	\end{stanza}
	
	\begin{stanza}
		\textsc{The waves} beside them danced; but they\verseline
		Out-did the sparkling waves in glee:\verseline
		A poet could not but be gay,\verseline
		In such a jocund company:\verseline
		I gazed—and gazed—but little thought\verseline
		What wealth the show to me had brought:
	\end{stanza}
	
	\begin{stanza}
		\textsc{For oft,} when on my couch I lie\verseline
		In vacant or in pensive mood,\verseline
		They flash upon that inward eye\verseline
		Which is the bliss of solitude;\verseline
		And then my heart with pleasure fills,\verseline
		And dances with the daffodils.
	\end{stanza}
\end{poem}

\pagebreak

\begin{volumetitlepage}
\sequencetitle{\textbf{III.} Autumn}
\bigskip
\noindent\fbox{
	\parbox{\textwidth}{
		\poemtitle{\textbf{The Road Not Taken}}
		\attribution{Robert Frost (26 Mar 1874 - 29 Jan 1963)}
		\begin{poem}
			\begin{stanza}
				\textsc{Two roads} diverged in a yellow wood,\verseline
				And sorry I could not travel both\verseline
				And be one traveler, long I stood\verseline
				And looked down one as far as I could\verseline
				To where it bent in the undergrowth;
			\end{stanza}
			
			\begin{stanza}
				\textsc{Then took} the other, as just as fair,\verseline
				And having perhaps the better claim,\verseline
				Because it was grassy and wanted wear;\verseline
				Though as for that the passing there\verseline
				Had worn them really about the same,
			\end{stanza}
			
			\begin{stanza}
				\textsc{And both} that morning equally lay\verseline
				In leaves no step had trodden black.\verseline
				Oh, I kept the first for another day!\verseline
				Yet knowing how way leads on to way,\verseline
				I doubted if I should ever come back.
			\end{stanza}
			
			\begin{stanza}
				\textsc{I shall} be telling this with a sigh\verseline
				Somewhere ages and ages hence:\verseline
				Two roads diverged in a wood, and I—\verseline
				I took the one less traveled by,\verseline
				And that has made all the difference.
			\end{stanza}
		\end{poem}
	}
}

\end{volumetitlepage}

\pagebreak

\afterpage{\blankpage}
\poemtitle{(Autumn 1) \textbf{\textit{All The World's a Stage}}}
\attribution{Jaques, \underline{As You Like It} by William Shakespeare (?? Apr 1564 - 23 Apr 1616)}
\begin{poem}
	\begin{stanza}
		\verseindent\verseindent\verseindent\textsc{All the} world’s a stage,\verseline
		And all the men and women merely players;\verseline
		They have their exits and their entrances;\verseline
		And one man in his time plays many parts,\verseline
		His acts being seven ages. At first the infant,\verseline
		Mewling and puking in the nurse’s arms;\verseline
		And then the whining school-boy, with his satchel\verseline
		And shining morning face, creeping like snail\verseline
		Unwillingly to school. And then the lover,\verseline
		Sighing like furnace, with a woeful ballad\verseline
		Made to his mistress’ eyebrow. Then a soldier,\verseline
		Full of strange oaths, and bearded like the pard,\verseline
		Jealous in honour, sudden and quick in quarrel,\verseline
		Seeking the bubble reputation\verseline
		Even in the cannon’s mouth. And then the justice,\verseline
		In fair round belly with good capon lin’d,\verseline
		With eyes severe and beard of formal cut,\verseline
		Full of wise saws and modern instances;\verseline
		And so he plays his part. The sixth age shifts\verseline
		Into the lean and slipper’d pantaloon,\verseline
		With spectacles on nose and pouch on side;\verseline
		His youthful hose, well sav’d, a world too wide\verseline
		For his shrunk shank; and his big manly voice,\verseline
		Turning again toward childish treble, pipes\verseline
		And whistles in his sound. Last scene of all,\verseline
		That ends this strange eventful history,\verseline
		Is second childishness and mere oblivion;\verseline
		Sans teeth, sans eyes, sans taste, sans everything.
	\end{stanza}
\end{poem}

\pagebreak

\poemtitle{(Autumn 2) \textbf{It Is Later Than You Think}}
\attribution{Robert Service (16 Jan 1874 - 11 Sep 1958)}
\begin{poem}
	\begin{stanza}
		\textsc{Lone amid} the café’s cheer,\verseline
		Sad of heart am I to-night;\verseline
		Dolefully I drink my beer,\verseline
		But no single line I write.\verseline
		There’s the wretched rent to pay,\verseline
		Yet I glower at pen and ink:\verseline
		Oh, inspire me, Muse, I pray,\verseline
		It is later than you think!
	\end{stanza}
	
	\begin{stanza}
		\textsc{Hello! there’s} a pregnant phrase.\verseline
		Bravo! let me write it down;\verseline
		Hold it with a hopeful gaze,\verseline
		Gauge it with a fretful frown;\verseline
		Tune it to my lyric lyre ...   \verseline
		Ah! upon starvation’s brink,\verseline
		How the words are dark and dire:\verseline
		It is later than you think.
	\end{stanza}
	
	\begin{stanza}
		\textsc{Weigh them} well .... Behold yon band,\verseline
		Students drinking by the door,\verseline
		Madly merry, bock in hand,\verseline
		Saucers stacked to mark their score.\verseline
		Get you gone, you jolly scamps;\verseline
		Let your parting glasses clink;\verseline
		Seek your long neglected lamps:\verseline
		It is later than you think.
	\end{stanza}
	
	\begin{stanza}
		\textsc{Look again:} yon dainty blonde,\verseline
		All allure and golden grace,\verseline
		Oh so willing to respond\verseline
		Should you turn a smiling face.\verseline
		Play your part, poor pretty doll;\verseline
		Feast and frolic, pose and prink;\verseline
		There’s the Morgue to end it all,\verseline
		And it’s later than you think.
	\end{stanza}
	
	\begin{stanza}
		\textsc{Yon’s a} playwright — mark his face,\verseline
		Puffed and purple, tense and tired;\verseline
		Pasha-like he holds his place,\verseline
		Hated, envied and admired.\verseline
		How you gobble life, my friend;\verseline
		Wine, and woman soft and pink!\verseline
		Well, each tether has its end:\verseline
		Sir, it’s later than you think.
	\end{stanza}
	
	\begin{stanza}
		\textsc{See yon} living scarecrow pass\verseline
		With a wild and wolfish stare\verseline
		At each empty absinthe glass,\verseline
		As if he saw Heaven there.\verseline
		Poor damned wretch, to end your pain\verseline
		There is still the Greater Drink.\verseline
		Yonder waits the sanguine Seine ...\verseline
		It is later than you think.
	\end{stanza}
	
	\begin{stanza}
		\textsc{Lastly, you} who read; aye, you\verseline
		Who this very line may scan:\verseline
		Think of all you planned to do ...   \verseline
		Have you done the best you can?\verseline
		See! the tavern lights are low;\verseline
		Black’s the night, and how you shrink!\verseline
		God! and is it time to go?\verseline
		Ah! the clock is always slow;\verseline
		It is later than you think;\verseline
		Sadly later than you think;\verseline
		Far, far later than you think.
	\end{stanza}
\end{poem}

\pagebreak

\afterpage{\blankpage}
\poemtitle{(Autumn 3) \textbf{A Lecture Upon the Shadow}}
\attribution{John Donne (22 Jan 1572 - 31 Mar 1631)}
\begin{poem}
	\begin{stanza}
		\textsc{Stand still,} and I will read to thee\verseline
		A lecture, love, in love's philosophy.\verseline
		         \verseindent These three hours that we have spent,\verseline
		         \verseindent Walking here, two shadows went\verseline
		Along with us, which we ourselves produc'd.\verseline
		But, now the sun is just above our head,\verseline
		         \verseindent We do those shadows tread,\verseline
		         \verseindent And to brave clearness all things are reduc'd.\verseline
		So whilst our infant loves did grow,\verseline
		Disguises did, and shadows, flow\verseline
		From us, and our cares; but now 'tis not so.\verseline
		That love has not attain'd the high'st degree,\verseline
		Which is still diligent lest others see.
	\end{stanza}
	
	\begin{stanza}
		\textsc{Except our} loves at this noon stay,\verseline
		We shall new shadows make the other way.\verseline
		         \verseindent As the first were made to blind\verseline
		         \verseindent Others, these which come behind\verseline
		Will work upon ourselves, and blind our eyes.\verseline
		If our loves faint, and westwardly decline,\verseline
		         \verseindent To me thou, falsely, thine,\verseline
		         \verseindent And I to thee mine actions shall disguise.\verseline
		The morning shadows wear away,\verseline
		But these grow longer all the day;\verseline
		But oh, love's day is short, if love decay.\verseline
		Love is a growing, or full constant light,\verseline
		And his first minute, after noon, is night.
	\end{stanza}
\end{poem}

\pagebreak

\poemtitle{(Autumn 4) \textbf{Sailing to Byzantium}}
\attribution{William Butler Yeats (13 Jun 1865 - 28 Jan 1939)}
\begin{poem}
	\begin{stanza}
		\textbf{I.}
		
		\textsc{That is} no country for old men. The young\verseline
		In one another's arms, birds in the trees,\verseline
		—Those dying generations—at their song,\verseline
		The salmon-falls, the mackerel-crowded seas,\verseline
		Fish, flesh, or fowl, commend all summer long\verseline
		Whatever is begotten, born, and dies.\verseline
		Caught in that sensual music all neglect\verseline
		Monuments of unageing intellect.
	\end{stanza}
	
	\begin{stanza}
		\textbf{II.}
		
		\textsc{An aged} man is but a paltry thing,\verseline
		A tattered coat upon a stick, unless\verseline
		Soul clap its hands and sing, and louder sing\verseline
		For every tatter in its mortal dress,\verseline
		Nor is there singing school but studying\verseline
		Monuments of its own magnificence;\verseline
		And therefore I have sailed the seas and come\verseline
		To the holy city of Byzantium.
	\end{stanza}
	
	\pagebreak

	\begin{stanza}
		\textbf{III.}
		
		\textsc{O sages} standing in God's holy fire\verseline
		As in the gold mosaic of a wall,\verseline
		Come from the holy fire, perne in a gyre,\verseline
		And be the singing-masters of my soul.\verseline
		Consume my heart away; sick with desire\verseline
		And fastened to a dying animal\verseline
		It knows not what it is; and gather me\verseline
		Into the artifice of eternity.
	\end{stanza}
	
	\begin{stanza}
		\textbf{IV.}
		
		\textsc{Once out} of nature I shall never take\verseline
		My bodily form from any natural thing,\verseline
		But such a form as Grecian goldsmiths make\verseline
		Of hammered gold and gold enamelling\verseline
		To keep a drowsy Emperor awake;\verseline
		Or set upon a golden bough to sing\verseline
		To lords and ladies of Byzantium\verseline
		Of what is past, or passing, or to come.
	\end{stanza}
\end{poem}

\pagebreak

\afterpage{\blankpage}
\poemtitle{(Autumn 5) \textbf{A Dream Within a Dream}}
\attribution{Edgar Allan Poe (19 Jan 1809 - 7 Oct 1849)}
\begin{poem}
	\begin{stanza}
		\textsc{Take this} kiss upon the brow!\verseline
		And, in parting from you now,\verseline
		Thus much let me avow —\verseline
		You are not wrong, who deem\verseline
		That my days have been a dream;\verseline
		Yet if hope has flown away\verseline
		In a night, or in a day,\verseline
		In a vision, or in none,\verseline
		Is it therefore the less gone? \verseline
		All that we see or seem\verseline
		Is but a dream within a dream.
	\end{stanza}
	
	\begin{stanza}
		\textsc{I stand} amid the roar\verseline
		Of a surf-tormented shore,\verseline
		And I hold within my hand\verseline
		Grains of the golden sand —\verseline
		How few! yet how they creep\verseline
		Through my fingers to the deep,\verseline
		While I weep — while I weep!\verseline
		O God! Can I not grasp\verseline
		Them with a tighter clasp?\verseline
		O God! can I not save\verseline
		One from the pitiless wave?\verseline
		Is all that we see or seem\verseline
		But a dream within a dream?
	\end{stanza}
\end{poem}

\pagebreak

\poemtitle{(Autumn 6) \textbf{Byzantium}}
\attribution{William Butler Yeats (13 Jun 1865 - 28 Jan 1939)}
\begin{poem}
	\begin{stanza}
		\textsc{The unpurged} images of day recede;\verseline
		The Emperor's drunken soldiery are abed;\verseline
		Night resonance recedes, night-walkers' song\verseline
		After great cathedral gong;\verseline
		A starlit or a moonlit dome disdains\verseline
		All that man is,\verseline
		All mere complexities,\verseline
		The fury and the mire of human veins.
	\end{stanza}
	
	\begin{stanza}
		\textsc{Before me} floats an image, man or shade,\verseline
		Shade more than man, more image than a shade;\verseline
		For Hades' bobbin bound in mummy-cloth\verseline
		May unwind the winding path;\verseline
		A mouth that has no moisture and no breath\verseline
		Breathless mouths may summon;\verseline
		I hail the superhuman;\verseline
		I call it death-in-life and life-in-death.
	\end{stanza}
	
	\pagebreak

	\begin{stanza}
		\textsc{Miracle, bird} or golden handiwork,\verseline
		More miracle than bird or handiwork,\verseline
		Planted on the starlit golden bough,\verseline
		Can like the cocks of Hades crow,\verseline
		Or, by the moon embittered, scorn aloud\verseline
		In glory of changeless metal\verseline
		Common bird or petal\verseline
		And all complexities of mire or blood.
	\end{stanza}
	
	\begin{stanza}
		\textsc{At midnight} on the Emperor's pavement flit\verseline
		Flames that no faggot feeds, nor steel has lit,\verseline
		Nor storm disturbs, flames begotten of flame,\verseline
		Where blood-begotten spirits come\verseline
		And all complexities of fury leave,\verseline
		Dying into a dance,\verseline
		An agony of trance,\verseline
		An agony of flame that cannot singe a sleeve.
	\end{stanza}
	
	\begin{stanza}
		\textsc{Astraddle on} the dolphin's mire and blood,\verseline
		Spirit after spirit! The smithies break the flood,\verseline
		The golden smithies of the Emperor!\verseline
		Marbles of the dancing floor\verseline
		Break bitter furies of complexity,\verseline
		Those images that yet\verseline
		Fresh images beget,\verseline
		That dolphin-torn, that gong-tormented sea.
	\end{stanza}
\end{poem}

\pagebreak

\begin{volumetitlepage}
\sequencetitle{\textbf{IV.} Winter}
\bigskip
\noindent\fbox{
	\parbox{\textwidth}{
		\poemtitle{\textbf{Ozymandias}}
		\attribution{Percy Bysshe Shelley (4 Aug 1792 - 8 Jul 1822)}
		\begin{poem}
			\begin{stanza}
				\textsc{I met} a traveller from an antique land,\verseline
				Who said—“Two vast and trunkless legs of stone\verseline
				Stand in the desert. . . . Near them, on the sand,\verseline
				Half sunk a shattered visage lies, whose frown,\verseline
				And wrinkled lip, and sneer of cold command,\verseline
				Tell that its sculptor well those passions read\verseline
				Which yet survive, stamped on these lifeless things,\verseline
				The hand that mocked them, and the heart that fed;\verseline
				And on the pedestal, these words appear:\verseline
				\verseindent 'My name is Ozymandias, King of Kings;\verseline
				\verseindent Look on my Works, ye Mighty, and despair!'\verseline
				Nothing beside remains. Round the decay\verseline
				Of that colossal Wreck, boundless and bare\verseline
				The lone and level sands stretch far away.”
			\end{stanza}
		\end{poem}
	}
}

\end{volumetitlepage}

\pagebreak

\afterpage{\blankpage}
\poemtitle{(Winter 1) \textbf{Lines Written in Early Spring}}
\attribution{William Wordsworth (7 Apr 1770 - 23 Apr 1850)}
\begin{poem}

	\begin{stanza}
		I heard a thousand blended notes,\verseline
		While in a grove I sate reclined,\verseline
		In that sweet mood when pleasant thoughts\verseline
		Bring sad thoughts to the mind.
	\end{stanza}
	
	\begin{stanza}
		To her fair works did Nature link\verseline
		The human soul that through me ran;\verseline
		And much it grieved my heart to think\verseline
		What man has made of man.
	\end{stanza}
	
	\begin{stanza}
		Through primrose tufts, in that green bower,\verseline
		The periwinkle trailed its wreaths;\verseline
		And ’tis my faith that every flower\verseline
		Enjoys the air it breathes.
	\end{stanza}
	
	\begin{stanza}
		The birds around me hopped and played,\verseline
		Their thoughts I cannot measure:—\verseline
		But the least motion which they made\verseline
		It seemed a thrill of pleasure.
	\end{stanza}
	
	\begin{stanza}
		The budding twigs spread out their fan,\verseline
		To catch the breezy air;\verseline
		And I must think, do all I can,\verseline
		That there was pleasure there.
	\end{stanza}
	
	\begin{stanza}
		If this belief from heaven be sent,\verseline
		If such be Nature’s holy plan,\verseline
		Have I not reason to lament\verseline
		What man has made of man?
	\end{stanza}
\end{poem}

\pagebreak

\afterpage{\blankpage}
\poemtitle{(Winter 2) \textbf{Time to Come}}
\attribution{Walt Whitman (31 May 1819 - 26 Mar 1892)}
\begin{poem}
	\begin{stanza}
		O, Death! a black and pierceless pall\verseline
		    \verseindent Hangs round thee, and the future state;\verseline
		No eye may see, no mind may grasp\verseline
		    \verseindent That mystery of fate.
	\end{stanza}
	
	\begin{stanza}
		This brain, which now alternate throbs\verseline
		    \verseindent With swelling hope and gloomy fear;\verseline
		This heart, with all the changing hues,\verseline
		    \verseindent That mortal passions bear—
	\end{stanza}
	
	\begin{stanza}
		This curious frame of human mould,\verseline
		    \verseindent Where unrequited cravings play,\verseline
		This brain, and heart, and wondrous form\verseline
		    \verseindent Must all alike decay.
	\end{stanza}
	
	\begin{stanza}
		The leaping blood will stop its flow;\verseline
		    \verseindent The hoarse death-struggle pass; the cheek\verseline
		Lay bloomless, and the liquid tongue\verseline
		    \verseindent Will then forget to speak.
	\end{stanza}
	
	\begin{stanza}
		The grave will take me; earth will close\verseline
		    \verseindent O’er cold dull limbs and ashy face;\verseline
		But where, O, Nature, where shall be\verseline
		    \verseindent The soul’s abiding place?
	\end{stanza}
	
	\begin{stanza}
		Will it e’en live? For though its light\verseline
		    \verseindent Must shine till from the body torn;\verseline
		Then, when the oil of life is spent,   \verseline
		    \verseindent Still shall the taper burn?
	\end{stanza}

	\begin{stanza}
		O, powerless is this struggling brain\verseline
		    \verseindent To rend the mighty mystery;\verseline
		In dark, uncertain awe it waits\verseline
		    \verseindent The common doom, to die.
	\end{stanza}
\end{poem}

\pagebreak

\afterpage{\blankpage}
\poemtitle{(Winter 3) \textbf{I am A Little World Cunningly Made}}
\attribution{John Donne (22 Jan 1572 - 31 Mar 1631)}
\begin{poem}

	\begin{stanza}
		I am a little world made cunningly\verseline
		Of elements and an angelic sprite,\verseline
		But black sin hath betray'd to endless night\verseline
		My world's both parts, and oh both parts must die.\verseline
		You which beyond that heaven which was most high\verseline
		Have found new spheres, and of new lands can write,\verseline
		Pour new seas in mine eyes, that so I might\verseline
		Drown my world with my weeping earnestly,\verseline
		Or wash it, if it must be drown'd no more.\verseline
		But oh it must be burnt; alas the fire\verseline
		Of lust and envy have burnt it heretofore,\verseline
		And made it fouler; let their flames retire,\verseline
		And burn me O Lord, with a fiery zeal\verseline
		Of thee and thy house, which doth in eating heal.
	\end{stanza}

\end{poem}

\pagebreak

\poemtitle{(Winter 4) \textbf{The Conqueror Worm}}
\attribution{Edgar Allan Poe (19 Jan 1809 - 7 Oct 1849)}
\begin{poem}
	\begin{stanza}
		Lo! ’t is a gala night\verseline
		   \verseindent Within the lonesome latter years!   \verseline
		An angel throng, bewinged, bedight\verseline
		   \verseindent In veils, and drowned in tears,   \verseline
		Sit in a theatre, to see\verseline
		   \verseindent A play of hopes and fears,\verseline
		While the orchestra breathes fitfully   \verseline
		   \verseindent The music of the spheres.
	\end{stanza}
	
	\begin{stanza}
		Mimes, in the form of God on high,   \verseline
		   \verseindent Mutter and mumble low,\verseline
		And hither and thither fly—\verseline
		   \verseindent Mere puppets they, who come and go   \verseline\
		At bidding of vast formless things\verseline
		   \verseindent That shift the scenery to and fro,\verseline
		Flapping from out their Condor wings\verseline
		   \verseindent Invisible Wo!
	\end{stanza}
	
	\begin{stanza}
		That motley drama—oh, be sure   \verseline
		  \verseindent It shall not be forgot!\verseline
		With its Phantom chased for evermore   \verseline
		  \verseindent By a crowd that seize it not,\verseline
		Through a circle that ever returneth in   \verseline
		   \verseindent To the self-same spot,\verseline
		And much of Madness, and more of Sin,   \verseline
		  \verseindent And Horror the soul of the plot.
	\end{stanza}

	\pagebreak	

	\begin{stanza}
		But see, amid the mimic rout,\verseline
		  \verseindent A crawling shape intrude!\verseline
		A blood-red thing that writhes from out   \verseline
		  \verseindent The scenic solitude!\verseline
		It writhes!—it writhes!—with mortal pangs   \verseline
		The mimes become its food,\verseline
		And seraphs sob at vermin fangs\verseline
		  \verseindent In human gore imbued.
	\end{stanza}
	
	\begin{stanza}
		Out—out are the lights—out all!   \verseline
		  \verseindent And, over each quivering form,\verseline
		The curtain, a funeral pall,\verseline
		   Comes down with the rush of a storm,   \verseline
		While the angels, all pallid and wan,   \verseline
		   \verseindent Uprising, unveiling, affirm\verseline
		That the play is the tragedy, “Man,”   \verseline
		   \verseindent And its hero, the Conqueror Worm.
	\end{stanza}
\end{poem}

\pagebreak

\afterpage{\blankpage}
\poemtitle{(Winter 5) \textbf{Do Not Go Gentle}}
\attribution{Dylan Thomas (27 October 1914 – 9 November 1953)}
\begin{poem}
	\begin{stanza}
		Do not go gentle into that good night,\verseline
		Old age should burn and rave at close of day;\verseline
		Rage, rage against the dying of the light.
	\end{stanza}
	
	\begin{stanza}
		Though wise men at their end know dark is right,\verseline
		Because their words had forked no lightning they\verseline
		Do not go gentle into that good night.
	\end{stanza}
	
	\begin{stanza}
		Good men, the last wave by, crying how bright\verseline
		Their frail deeds might have danced in a green bay,\verseline
		Rage, rage against the dying of the light.
	\end{stanza}
	
	\begin{stanza}
		Wild men who caught and sang the sun in flight,\verseline
		And learn, too late, they grieved it on its way,\verseline
		Do not go gentle into that good night.
	\end{stanza}
	
	\begin{stanza}
		Grave men, near death, who see with blinding sight\verseline
		Blind eyes could blaze like meteors and be gay,\verseline
		Rage, rage against the dying of the light.
	\end{stanza}
	
	\begin{stanza}
		And you, my father, there on the sad height,\verseline
		Curse, bless, me now with your fierce tears, I pray.\verseline
		Do not go gentle into that good night.\verseline
		Rage, rage against the dying of the light.
	\end{stanza}
\end{poem}

\pagebreak

\poemtitle{(Winter 6) \textbf{\textit{To Be or Not To Be}}}
\attribution{Prince Hamlet, \underline{Hamlet} \\ by William Shakespeare (?? Apr 1564 - 23 Apr 1616)}
\begin{poem}
	\begin{stanza}
		To be, or not to be, that is the question:\verseline
		Whether 'tis nobler in the mind to suffer\verseline
		The slings and arrows of outrageous fortune,\verseline
		Or to take arms against a sea of troubles\verseline
		And by opposing end them. To die—to sleep,\verseline
		No more; and by a sleep to say we end\verseline
		The heart-ache and the thousand natural shocks\verseline
		That flesh is heir to: 'tis a consummation\verseline
		Devoutly to be wish'd. To die, to sleep;\verseline
		To sleep, perchance to dream—ay, there's the rub:\verseline
		For in that sleep of death what dreams may come,\verseline
		When we have shuffled off this mortal coil,\verseline
		Must give us pause—there's the respect\verseline
		That makes calamity of so long life.\verseline
		For who would bear the whips and scorns of time,\verseline
		Th'oppressor's wrong, the proud man's contumely,\verseline
		The pangs of dispriz'd love, the law's delay,\verseline
		The insolence of office, and the spurns\verseline
		That patient merit of th'unworthy takes,\verseline
		When he himself might his quietus make\verseline
		With a bare bodkin? Who would fardels bear,\verseline
		To grunt and sweat under a weary life,\verseline
		But that the dread of something after death,\verseline
		The undiscovere'd country, from whose bourn\verseline
		No traveller returns, puzzles the will,\verseline
		And makes us rather bear those ills we have\verseline
		Than fly to others that we know not of?\verseline
		Thus conscience doth make cowards of us all,\verseline
		And thus the native hue of resolution\verseline
		Is sicklied o'er with the pale cast of thought,\verseline
		And enterprises of great pith and moment\verseline
		With this regard their currents turn awry\verseline
		And lose the name of action.
	\end{stanza}
\end{poem}

\begin{volumetitlepage}
\sequencetitle{\textbf{V.} A Season Out Of Time}
\bigskip
\noindent\fbox{
	\parbox{\textwidth}{
		\poemtitle{\textbf{Death, Be Not Proud}}
		\attribution{Walt Whitman (31 May 1819 - 26 Mar 1892)}
		\begin{poem}
			\begin{stanza}
				Death, be not proud, though some have called thee\verseline
				Mighty and dreadful, for thou art not so;\verseline
				For those whom thou think'st thou dost overthrow\verseline
				Die not, poor Death, nor yet canst thou kill me.\verseline
				From rest and sleep, which but thy pictures be,\verseline
				Much pleasure; then from thee much more must flow,\verseline
				And soonest our best men with thee do go,\verseline
				Rest of their bones, and soul's delivery.\verseline
				Thou art slave to fate, chance, kings, and desperate men,\verseline
				And dost with poison, war, and sickness dwell,\verseline
				And poppy or charms can make us sleep as well\verseline
				And better than thy stroke; why swell'st thou then?\verseline
				One short sleep past, we wake eternally\verseline
				And death shall be no more; Death, thou shalt die. 
			\end{stanza}
		\end{poem}
	}
}

\end{volumetitlepage} 

\pagebreak

\afterpage{\blankpage}
\poemtitle{(Epilogue 1) \textbf{O Captain! My Captain!}}
\attribution{Walt Whitman (31 May 1819 - 26 Mar 1892)}
\begin{poem}
	\begin{stanza}
		O Captain! my Captain! our fearful trip is done,\verseline
		The ship has weather’d every rack, the prize we sought is won,\verseline
		The port is near, the bells I hear, the people all exulting,\verseline
		While follow eyes the steady keel, the vessel grim and daring;\verseline
		                         \verseindent But O heart! heart! heart!\verseline
		                          \verseindent \verseindent O the bleeding drops of red,\verseline
		                           \verseindent \verseindent \verseindent Where on the deck my Captain lies,\verseline
		                                 \verseindent \verseindent \verseindent \verseindent Fallen cold and dead.
	\end{stanza}
	
	\begin{stanza}
		O Captain! my Captain! rise up and hear the bells;\verseline
		Rise up—for you the flag is flung—for you the bugle trills,\verseline
		For you bouquets and ribbon’d wreaths—for you the shores a-crowding,\verseline
		For you they call, the swaying mass, their eager faces turning;\verseline
		                         \verseindent Here Captain! dear father!\verseline
		                            \verseindent \verseindent This arm beneath your head!\verseline
		                               \verseindent \verseindent \verseindent It is some dream that on the deck,\verseline
		                                 \verseindent \verseindent \verseindent \verseindent You’ve fallen cold and dead.
	\end{stanza}
	
	\begin{stanza}
		My Captain does not answer, his lips are pale and still,\verseline
		My father does not feel my arm, he has no pulse nor will,\verseline
		The ship is anchor’d safe and sound, its voyage closed and done,\verseline
		From fearful trip the victor ship comes in with object won;\verseline
		                         \verseindent Exult O shores, and ring O bells!\verseline
		                            \verseindent \verseindent But I with mournful tread,\verseline
		                               \verseindent \verseindent \verseindent Walk the deck my Captain lies,\verseline
		                                  \verseindent \verseindent \verseindent \verseindent Fallen cold and dead.
	\end{stanza}
\end{poem}

\pagebreak


\poemtitle{(Epilogue 2) \textbf{The Ballad of God-Makers}}
\attribution{G.K. Chesterton (29 May 1874 – 14 Jun 1936)}
\begin{poem}
	\begin{stanza}
		A bird flew out at the break of day\verseline
		From the nest where it had curled,\verseline
		And ere the eve the bird had set\verseline
		Fear on the kings of the world.
	\end{stanza}
	
	\begin{stanza}
		The first tree it lit upon\verseline
		Was green with leaves unshed;\verseline
		The second tree it lit upon\verseline
		Was red with apples red;
	\end{stanza}
	
	\begin{stanza}
		The third tree it lit upon\verseline
		Was barren and was brown,\verseline
		Save for a dead man nailed thereon\verseline
		On a hill above a town.
	\end{stanza}
	
	\begin{stanza}
		That night the kings of the earth were gay\verseline
		And filled the cup and can;\verseline
		Last night the kings of the earth were chill\verseline
		For dread of a naked man.
	\end{stanza}
	
	\begin{stanza}
		‘If he speak two more words,’ they said,\verseline
		‘The slave is more than the free;\verseline
		If he speak three more words,’ they said,\verseline
		‘The stars are under the sea.’
	\end{stanza}
	
	\begin{stanza}
		Said the King of the East to the King of the West,\verseline
		I wot his frown was set,\verseline
		‘Lo, let us slay him and make him as dung,\verseline
		It is well that the world forget.’
	\end{stanza}
	
	\begin{stanza}
		Said the King of the West to the King of the East,\verseline
		I wot his smile was dread,\verseline
		‘Nay, let us slay him and make him a god,\verseline
		It is well that our god be dead.’
	\end{stanza}
	
	\begin{stanza}
		They set the young man on a hill,\verseline
		They nailed him to a rod;\verseline
		And there in darkness and in blood\verseline
		They made themselves a god.
	\end{stanza}
	
	\begin{stanza}
		And the mightiest word was left unsaid,\verseline
		And the world had never a mark,\verseline
		And the strongest man of the sons of men\verseline
		Went dumb into the dark.
	\end{stanza}
	
	\begin{stanza}
		Then hymns and harps of praise they brought,\verseline
		Incense and gold and myrrh,\verseline
		And they thronged above the seraphim,\verseline
		The poor dead carpenter.
	\end{stanza}
	
	\begin{stanza}
		‘Thou art the prince of all,’ they sang,\verseline
		‘Ocean and earth and air.’\verseline
		Then the bird flew on to the cruel cross,\verseline
		And hid in the dead man’s hair.
	\end{stanza}
	
	\begin{stanza}
		‘Thou art the son of the world.’ they cried, `\verseline
		‘Speak if our prayers be heard.’\verseline
		And the brown bird stirred in the dead man’s hair\verseline
		And it seemed that the dead man stirred.
	\end{stanza}
	
	\begin{stanza}
		Then a shriek went up like the world’s last cry\verseline
		From all nations under heaven,\verseline
		And a master fell before a slave\verseline
		And begged to be forgiven.
	\end{stanza}
	
	\begin{stanza}
		They cowered, for dread in his wakened eyes\verseline
		The ancient wrath to see;\verseline
		And a bird flew out of the dead Christ’s hair,\verseline
		And lit on a lemon tree.
	\end{stanza}
\end{poem}

 
\pagebreak

\poemtitle{(Epilogue 3) \textbf{The Dead Man Walking}}
\attribution{Thomas Hardy (2 Jun 1840 - 11 Jan 1928)}
\begin{poem}
	\begin{stanza}
		They hail me as one living,\verseline
		But don't they know\verseline
		That I have died of late years,\verseline
		Untombed although?
	\end{stanza}
	
	\begin{stanza}
		I am but a shape that stands here,\verseline
		A pulseless mould,\verseline
		A pale past picture, screening\verseline
		Ashes gone cold.
	\end{stanza}
	
	\begin{stanza}
		Not at a minute's warning,\verseline
		Not in a loud hour,\verseline
		For me ceased Time's enchantments\verseline
		In hall and bower.
	\end{stanza}
	
	\begin{stanza}
		There was no tragic transit,\verseline
		No catch of breath,\verseline
		When silent seasons inched me\verseline
		On to this death ....
	\end{stanza}
	
	\begin{stanza}
		— A Troubadour-youth I rambled\verseline
		With Life for lyre,\verseline
		The beats of being raging\verseline
		In me like fire.
	\end{stanza}

	\pagebreak

	\begin{stanza}
		But when I practised eyeing\verseline
		The goal of men,\verseline
		It iced me, and I perished\verseline
		A little then.
	\end{stanza}
	
	\begin{stanza}
		When passed my friend, my kinsfolk,\verseline
		Through the Last Door,\verseline
		And left me standing bleakly,\verseline
		I died yet more;
	\end{stanza}
	
	\begin{stanza}
		And when my Love's heart kindled\verseline
		In hate of me,\verseline
		Wherefore I knew not, died I\verseline
		One more degree.
	\end{stanza}
	
	\begin{stanza}
		And if when I died fully\verseline
		I cannot say,\verseline
		And changed into the corpse-thing\verseline
		I am to-day,
	\end{stanza}
	
	\begin{stanza}
		Yet is it that, though whiling\verseline
		The time somehow\verseline
		In walking, talking, smiling,\verseline
		I live not now. 
	\end{stanza}
\end{poem}

\pagebreak

\afterpage{\blankpage}
\poemtitle{(Epilogue 4) \textbf{Stopping by Woods on a Snowy Night}}
\attribution{Robert Frost (26 Mar 1874 - 29 Jan 1963)}
\begin{poem}
	\begin{stanza}
		Whose woods these are I think I know.  \verseline
		His house is in the village though;   \verseline
		He will not see me stopping here   \verseline
		To watch his woods fill up with snow.   
	\end{stanza}
	
	\begin{stanza}
		My little horse must think it queer   \verseline
		To stop without a farmhouse near   \verseline
		Between the woods and frozen lake   \verseline
		The darkest evening of the year.   
	\end{stanza}
	
	\begin{stanza}
		He gives his harness bells a shake   \verseline
		To ask if there is some mistake.   \verseline
		The only other sound’s the sweep   \verseline
		Of easy wind and downy flake.   
	\end{stanza}
	
	\begin{stanza}
		The woods are lovely, dark and deep,   \verseline
		But I have promises to keep,   \verseline
		And miles to go before I sleep,   \verseline
		And miles to go before I sleep.
	\end{stanza}
\end{poem}

\pagebreak

\poemtitle{(Epilogue 5) \textbf{The Love Song of J. Alfred Prufrock}}
\attribution{T.S. Eliot (26 Sep 1888 - 4 Jan 1965)}
\begin{poem}

	\begin{indentedverse}
		\begin{stanza}
			\verseindent\verseindent\verseindent \textit{S’io credesse che mia risposta fosse}\verseline
			\textit{A persona che mai tornasse al mondo,}\verseline
			\textit{Questa fiamma staria senza piu scosse.}\verseline
			\textit{Ma percioche giammai di questo fondo}\verseline
			\textit{Non torno vivo alcun, s’i’odo il vero,}\verseline
			\textit{Senza tema d’infamia ti rispondo.}
		\end{stanza}
	\end{indentedverse}


	\begin{stanza}
		Let us go then, you and I,\verseline
		When the evening is spread out against the sky\verseline
		Like a patient etherized upon a table;\verseline
		Let us go, through certain half-deserted streets,\verseline
		The muttering retreats\verseline
		Of restless nights in one-night cheap hotels\verseline
		And sawdust restaurants with oyster-shells:\verseline
		Streets that follow like a tedious argument\verseline
		Of insidious intent\verseline
		To lead you to an overwhelming question ...\verseline
		Oh, do not ask, “What is it?”\verseline
		Let us go and make our visit.
	\end{stanza}
	
	\begin{stanza}
		In the room the women come and go\verseline
		Talking of Michelangelo.
	\end{stanza}
	
	\begin{stanza}
		The yellow fog that rubs its back upon the window-panes,\verseline
		The yellow smoke that rubs its muzzle on the window-panes,\verseline
		Licked its tongue into the corners of the evening,\verseline
		Lingered upon the pools that stand in drains,\verseline
		Let fall upon its back the soot that falls from chimneys,\verseline
		Slipped by the terrace, made a sudden leap,\verseline
		And seeing that it was a soft October night,\verseline
		Curled once about the house, and fell asleep.
	\end{stanza}

	\pagebreak
	
	\begin{stanza}
		And indeed there will be time\verseline
		For the yellow smoke that slides along the street,\verseline
		Rubbing its back upon the window-panes;\verseline
		There will be time, there will be time\verseline
		To prepare a face to meet the faces that you meet;\verseline
		There will be time to murder and create,\verseline
		And time for all the works and days of hands\verseline
		That lift and drop a question on your plate;\verseline
		Time for you and time for me,\verseline
		And time yet for a hundred indecisions,\verseline
		And for a hundred visions and revisions,\verseline
		Before the taking of a toast and tea.
	\end{stanza}
	
	\begin{stanza}
		In the room the women come and go\verseline
		Talking of Michelangelo.
	\end{stanza}
	
	\begin{stanza}
		And indeed there will be time\verseline
		To wonder, “Do I dare?” and, “Do I dare?”\verseline
		Time to turn back and descend the stair,\verseline
		With a bald spot in the middle of my hair —\verseline
		(They will say: “How his hair is growing thin!”)\verseline
		My morning coat, my collar mounting firmly to the chin,\verseline
		My necktie rich and modest, but asserted by a simple pin —\verseline
		(They will say: “But how his arms and legs are thin!”)\verseline
		Do I dare\verseline
		Disturb the universe?\verseline
		In a minute there is time\verseline
		For decisions and revisions which a minute will reverse.
	\end{stanza}
	
	\begin{stanza}
		For I have known them all already, known them all:\verseline
		Have known the evenings, mornings, afternoons,\verseline
		I have measured out my life with coffee spoons;\verseline
		I know the voices dying with a dying fall\verseline
		Beneath the music from a farther room.\verseline
		               \verseindent So how should I presume?
	\end{stanza}

	\pagebreak

	\begin{stanza}
		And I have known the eyes already, known them all—\verseline
		The eyes that fix you in a formulated phrase,\verseline
		And when I am formulated, sprawling on a pin,\verseline
		When I am pinned and wriggling on the wall,\verseline
		Then how should I begin\verseline
		To spit out all the butt-ends of my days and ways?\verseline
		               \verseindent And how should I presume?
	\end{stanza}
	
	\begin{stanza}
		And I have known the arms already, known them all—\verseline
		Arms that are braceleted and white and bare\verseline
		(But in the lamplight, downed with light brown hair!)\verseline
		Is it perfume from a dress\verseline
		That makes me so digress?\verseline
		Arms that lie along a table, or wrap about a shawl.\verseline
		               \verseindent And should I then presume?\verseline
		               \verseindent And how should I begin?
	\end{stanza}

	\begin{stanza}
		Shall I say, I have gone at dusk through narrow streets\verseline
		And watched the smoke that rises from the pipes\verseline
		Of lonely men in shirt-sleeves, leaning out of windows? ...
	\end{stanza}

	\begin{stanza}
		I should have been a pair of ragged claws\verseline
		Scuttling across the floors of silent seas.
	\end{stanza}
	
	\begin{stanza}
		And the afternoon, the evening, sleeps so peacefully!\verseline
		Smoothed by long fingers,\verseline
		Asleep ... tired ... or it malingers,\verseline
		Stretched on the floor, here beside you and me.\verseline
		Should I, after tea and cakes and ices,\verseline
		Have the strength to force the moment to its crisis?\verseline
		But though I have wept and fasted, wept and prayed,\verseline
		Though I have seen my head (grown slightly bald) \verseline
			\verseindent brought in upon a platter,\verseline
		I am no prophet — and here’s no great matter;\verseline
		I have seen the moment of my greatness flicker,\verseline
		And I have seen the eternal Footman hold my coat, \verseline
			\verseindent and snicker,\verseline
		And in short, I was afraid.
	\end{stanza}

	\pagebreak
	
	\begin{stanza}
		And would it have been worth it, after all,\verseline
		After the cups, the marmalade, the tea,\verseline
		Among the porcelain, among some talk of you and me,\verseline
		Would it have been worth while,\verseline
		To have bitten off the matter with a smile,\verseline
		To have squeezed the universe into a ball\verseline
		To roll it towards some overwhelming question,\verseline
		To say: “I am Lazarus, come from the dead,\verseline
		Come back to tell you all, I shall tell you all”—\verseline
		If one, settling a pillow by her head\verseline
		               \verseindent Should say: “That is not what I meant at all;\verseline
		               \verseindent That is not it, at all.”
	\end{stanza}

	\begin{stanza}
		And would it have been worth it, after all,\verseline
		Would it have been worth while,\verseline
		After the sunsets and the dooryards and the sprinkled streets,\verseline
		After the novels, after the teacups, \verseline
			\verseindent after the skirts that trail along the floor—\verseline
		And this, and so much more?—\verseline
		It is impossible to say just what I mean!\verseline
		But as if a magic lantern threw the nerves in patterns \verseline
			\verseindent on a screen:\verseline
		Would it have been worth while\verseline
		If one, settling a pillow or throwing off a shawl,\verseline
		And turning toward the window, should say:\verseline
		               \verseindent “That is not it at all,\verseline
		               \verseindent That is not what I meant, at all.”
	\end{stanza}
	
	\pagebreak

	\begin{stanza}
		No! I am not Prince Hamlet, nor was meant to be;\verseline
		Am an attendant lord, one that will do\verseline
		To swell a progress, start a scene or two,\verseline
		Advise the prince; no doubt, an easy tool,\verseline
		Deferential, glad to be of use,\verseline
		Politic, cautious, and meticulous;\verseline
		Full of high sentence, but a bit obtuse;\verseline
		At times, indeed, almost ridiculous—\verseline
		Almost, at times, the Fool.
	\end{stanza}
	
	\begin{stanza}
		I grow old ... I grow old ...\verseline
		I shall wear the bottoms of my trousers rolled.
	\end{stanza}
	
	\begin{stanza}
		Shall I part my hair behind?   Do I dare to eat a peach?\verseline
		I shall wear white flannel trousers, and walk upon the beach.\verseline
		I have heard the mermaids singing, each to each.
	\end{stanza}
	
	\begin{stanza}
		I do not think that they will sing to me.
	\end{stanza}
	
	\begin{stanza}
		I have seen them riding seaward on the waves\verseline
		Combing the white hair of the waves blown back\verseline
		When the wind blows the water white and black.\verseline
		We have lingered in the chambers of the sea\verseline
		By sea-girls wreathed with seaweed red and brown\verseline
		Till human voices wake us, and we drown.
	\end{stanza}
\end{poem}
	\afterpage{\blankpage}
\pagebreak

\afterpage{\blankpage}

\poemtitle{(Epilogue 6) \textbf{The Triple Fool}}
\attribution{John Donne (22 Jan 1572 - 31 Mar 1631)}
\begin{poem}

	\begin{stanza}
		\textsc{I am} two fools, I know,\verseline
		      \verseindent For loving, and for saying so\verseline
		         \verseindent\verseindent  In whining poetry;\verseline
		But where's that wiseman, that would not be I,\verseline
		          \verseindent If she would not deny?\verseline
		Then as th' earth's inward narrow crooked lanes\verseline
		    \verseindent Do purge sea water's fretful salt away,\verseline
		I thought, if I could draw my pains\verseline
		    \verseindent Through rhyme's vexation, I should them allay.\verseline
		Grief brought to numbers cannot be so fierce,\verseline
		For he tames it, that fetters it in verse.
	\end{stanza}
	
	\begin{stanza}
		      \verseindent \textsc{But when} I have done so,\verseline
		      \verseindent Some man, his art and voice to show,\verseline
		          \verseindent\verseindent Doth set and sing my pain;\verseline
		And, by delighting many, frees again\verseline
		          \verseindent Grief, which verse did restrain.\verseline
		To love and grief tribute of verse belongs,\verseline
		    \verseindent But not of such as pleases when 'tis read.\verseline
		Both are increased by such songs,\verseline
		    \verseindent For both their triumphs so are published,\verseline
		And I, which was two fools, do so grow three;\verseline
		Who are a little wise, the best fools be. 
	\end{stanza}
\end{poem}

\pagebreak


\begin{volumetitlepage}

\huge{Fin...?}

\end{volumetitlepage}

\pagebreak
\begin{volumetitlepage}
	\poemtitle{\textbf{Dies Irae}}
	\begin{poem}
		\begin{stanza}
			\textbf{I.}\verseline
			
			\textsc{Day of} wrath and doom impending!\verseline
			David's word with Sybil's blending,\verseline
			Heaven and earth in ashes ending!
		\end{stanza}
	
		\begin{stanza}
			\textbf{II.}\verseline
			
			\textsc{Oh, what} fear man's bosom rendeth,\verseline
			When from heaven the Judge descendeth,\verseline
			On whose sentence all dependeth.
		\end{stanza}
	
		\begin{stanza}
			\textbf{III.}\verseline
			
			\textsc{Wondrous sound} the trumpet flingeth;\verseline
			Through earth's sepulchres it ringeth;\verseline
			All before the throne it bringeth.
		\end{stanza}
	
		\begin{stanza}
			\textbf{IV.}\verseline
			
			\textsc{Death is} struck. and nature quaking.\verseline
			All creation is awaking,\verseline
			To its Judge an answer making.
		\end{stanza}

		\begin{stanza}
			\textbf{V.}\verseline
			
			\textsc{Lo, the} book, exactly worded,\verseline
			Wherein all hath been recorded,\verseline
			Thence shall judgement be awarded. 
		\end{stanza}
	
		\begin{stanza}
			\textbf{VI.}\verseline
			
			\textsc{When the} Judge his seat attaineth,\verseline
			And each hidden deed arraigneth,\verseline
			Nothing unavenged remaineth. 
		\end{stanza}

		\pagebreak
	
		\begin{stanza}
			\textbf{VII.}\verseline
			
			\textsc{What shall} I, frail man, be pleading?\verseline
			Who for me be interceding,\verseline
			When the just are mercy needing? 
		\end{stanza}
	
		\begin{stanza}
			\textbf{VIII.}\verseline
			
			\textsc{King of} Majesty tremendous,\verseline
			Who dost free salvation send us,\verseline
			Fount of pity, then befriend us! 
		\end{stanza}
	
		\begin{stanza}
			\textbf{IX.}\verseline
			
			\textsc{Think, kind} Jesu!—my salvation\verseline
			Caused Thy wondrous Incarnation;\verseline
			Leave me not to reprobation. 
		\end{stanza}
	
		\begin{stanza}
			\textbf{X.}\verseline
			
			\textsc{Faint and} weary, Thou hast sought me,\verseline
			On the Cross of suffering bought me.\verseline
			Shall such grace be vainly brought me? 
		\end{stanza}
	
		\begin{stanza}
			\textbf{XI.}\verseline
			
			\textsc{Righteous Judge,} for sin's pollution\verseline
			Grant Thy gift of absolution,\verseline
			Ere the day of retribution. 
		\end{stanza}
	
		\begin{stanza}
			\textbf{XII.}\verseline
			
			\textsc{Guilty, now} I pour my moaning,\verseline
			All my shame with anguish owning;\verseline
			Spare, O God, Thy suppliant groaning! 
		\end{stanza}
	
		\begin{stanza}
			\textbf{XIII.}\verseline
			
			\textsc{Through the} sinful woman shriven,\verseline
			Through the dying thief forgiven,\verseline
			Thou to me a hope hast given. 
		\end{stanza}
		
		\pagebreak
	
		\begin{stanza}
			\textbf{XIV.}\verseline
			
			\textsc{Worthless are} my prayers and sighing,\verseline
			Yet, good Lord, in grace complying,\verseline
			Rescue me from fires undying. 
		\end{stanza}
	
		\begin{stanza}
			\textbf{XV.}\verseline
			
			\textsc{With Thy} sheep a place provide me,\verseline
			From the goats afar divide me,\verseline
			To Thy right hand do Thou guide me. 
		\end{stanza}
	
		\begin{stanza}
			\textbf{XVI.}\verseline
			
			\textsc{When the} wicked are confounded,\verseline
			Doomed to flames of woe unbounded,\verseline
			Call me with Thy saints surrounded. 
		\end{stanza}
	
		\begin{stanza}
			\textbf{XVII.}\verseline
			
			\textsc{Low I} kneel, with heart's submission,\verseline
			See, like ashes, my contrition,\verseline
			Help me in my last condition. 
		\end{stanza}
	
		\begin{stanza}
			\textbf{XVIII.}\verseline
			
			\textsc{Ah! that} day of tears and mourning,\verseline
			From the dust of earth returning\verseline
			Man for judgement must prepare him,\verseline
			Spare, O God, in mercy spare him. 
		\end{stanza}
	
		\begin{stanza}
			\textbf{XIX.}\verseline
			
			\textsc{Lord, all-pitying,} Jesus blest,\verseline
			Grant them Thine eternal rest. Amen. 
		\end{stanza}
	\end{poem}
\end{volumetitlepage}

\end{document}

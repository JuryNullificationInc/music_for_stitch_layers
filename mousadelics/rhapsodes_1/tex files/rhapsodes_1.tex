\documentclass[fontsize=9, a5paper]{scrbook}
\usepackage{fancyhdr}
\usepackage[a5paper]{geometry}
\usepackage{ifthen}
\usepackage{keyval}
\usepackage{multicol}
\usepackage{makeidx}
\usepackage{mparhack}
\usepackage{poemscol}
\usepackage{afterpage}

\newcommand\blankpage{%
    \null
    \thispagestyle{empty}%
    \addtocounter{page}{-1}%
    \newpage}
\fancyhf{}
\cfoot{\thepage}
\pagestyle{fancy}
\stanzaatbottom{*}
\indexingontrue
\begin{document}

\begin{maintitlepage}
\wholebooktitle{\textbf{I.} A Volume of Stitched Lays}
\noindent\fbox{
	\parbox{\textwidth}{
		\poemtitle{The Human Seasons}
		\attribution{John Keats}
		\index{The Human Seasons}
		\begin{poem}
		    \begin{stanza}
		        \textsc{Four Seasons} fill the measure of the year;\verseline
			\index{\textsc{Four Seasons} fill the measure of the year;}
		        \verseindent There are four seasons in the mind of man:\verseline
		        He has his lusty Spring, when fancy clear\verseline
		             \verseindent Takes in all beauty with an easy span: \verseline
		        He has his Summer, when luxuriously \verseline
		             \verseindent Spring's honied cud of youthful thought he loves \verseline
		        To ruminate, and by such dreaming high \verseline
		             \verseindent Is nearest unto heaven: quiet coves \verseline
		        His soul has in its Autumn, when his wings \verseline
		             \verseindent He furleth close; contented so to look \verseline
		        On mists in idleness—to let fair things \verseline
		             \verseindent Pass by unheeded as a threshold brook. \verseline
		        He has his Winter too of pale misfeature, \verseline
		        Or else he would forego his mortal nature.
 		   \end{stanza}
		\end{poem}
	}
}
\end{maintitlepage}

\pmclcontentsname={Rhapsodies}
\putpoemcontents
\makepoemcontents

\pagenumbering{arabic}

\afterpage{\blankpage}
\begin{volumetitlepage}
\sequencetitle{\textbf{I.} Spring}
\bigskip
\noindent\fbox{
	\parbox{\textwidth}{
		\poemtitle{(Cantrip) \textbf{A Poet! He Hath Put His Heart to School!}}
		\attribution{William Wordsworth}
			\begin{poem}
\begin{stanza}
			 A poet!—He hath put his heart to school,\\
			Nor dares to move unpropped upon the staff\\
			Which art hath lodged within his hand—must laugh\\
			By precept only, and shed tears by rule.\\
			Thy Art be Nature; the live current quaff,\\
			And let the groveller sip his stagnant pool,\\
			In fear that else, when Critics grave and cool\\
			Have killed him, Scorn should write his epitaph.\\
			How does the Meadow-flower its bloom unfold?\\
			Because the lovely little flower is free\\
			Down to its root, and, in that freedom, bold;\\
			And so the grandeur of the Forest-tree\\
			Comes not by casting in a formal mould,\\
			But from its own divine vitality.
\end{stanza}
			\end{poem}
	}
}

\end{volumetitlepage}

\pagebreak

\poemtitle{(Spring 1) \textbf{The Bells}}
\attribution{Edgar Allan Poe (19 Jan 1809 - 7 Oct 1849)}
\begin{poem}
\begin{stanza}
\textbf{I.}

        \verseindent Hear the sledges with the bells—\\
                 \verseindent\verseindent\verseindent  Silver bells!\\
What a world of merriment their melody foretells!\\
        \verseindent How they tinkle, tinkle, tinkle,\\
           \verseindent\verseindent In the icy air of night!\\
        \verseindent While the stars that oversprinkle\\
        \verseindent All the heavens, seem to twinkle\\
           \verseindent\verseindent With a crystalline delight;\\
         \verseindent Keeping time, time, time,\\
         \verseindent In a sort of Runic rhyme,\\
To the tintinabulation that so musically wells\\
       \verseindent From the bells, bells, bells, bells,\\
               \verseindent\verseindent\verseindent Bells, bells, bells—\\
 From the jingling and the tinkling of the bells.
\end{stanza}

\pagebreak 
\begin{stanza}
\textbf{II.}

        \verseindent Hear the mellow wedding bells,\\
                 \verseindent\verseindent\verseindent  Golden bells!\\
What a world of happiness their harmony foretells!\\
        \verseindent Through the balmy air of night\\
        \verseindent How they ring out their delight!\\
           \verseindent\verseindent From the molten-golden notes,\\
               \verseindent\verseindent\verseindent And all in tune,\\
           \verseindent \verseindent What a liquid ditty floats\\
    \verseindent To the turtle-dove that listens, while she gloats\\
               \verseindent \verseindent\verseindent On the moon!\\
         \verseindent Oh, from out the sounding cells,\\
What a gush of euphony voluminously wells!\\
               \verseindent\verseindent\verseindent How it swells!\\
               \verseindent\verseindent\verseindent How it dwells\\
           \verseindent\verseindent On the Future! how it tells\\
           \verseindent\verseindent Of the rapture that impels\\
         \verseindent To the swinging and the ringing\\
           \verseindent\verseindent\verseindent Of the bells, bells, bells,\\
         \verseindent\verseindent Of the bells, bells, bells, bells,\\
               \verseindent\verseindent\verseindent Bells, bells, bells—\\
  To the rhyming and the chiming of the bells!
\end{stanza}
\pagebreak
\begin{stanza}
III.

         Hear the loud alarum bells—
                 Brazen bells!
What tale of terror, now, their turbulency tells!
       In the startled ear of night
       How they scream out their affright!
         Too much horrified to speak,
         They can only shriek, shriek,
                  Out of tune,
In a clamorous appealing to the mercy of the fire,
In a mad expostulation with the deaf and frantic fire,
            Leaping higher, higher, higher,
            With a desperate desire,
         And a resolute endeavor
         Now—now to sit or never,
       By the side of the pale-faced moon.
            Oh, the bells, bells, bells!
            What a tale their terror tells
                  Of Despair!
       How they clang, and clash, and roar!
       What a horror they outpour
On the bosom of the palpitating air!
       Yet the ear it fully knows,
            By the twanging,
            And the clanging,
         How the danger ebbs and flows;
       Yet the ear distinctly tells,
            In the jangling,
            And the wrangling.
       How the danger sinks and swells,
By the sinking or the swelling in the anger of the bells—
             Of the bells—
     Of the bells, bells, bells, bells,
            Bells, bells, bells—
 In the clamor and the clangor of the bells!
\end{stanza}
\pagebreak
\begin{stanza}
IV.

          Hear the tolling of the bells—
                 Iron bells!
What a world of solemn thought their monody compels!
        In the silence of the night,
        How we shiver with affright
  At the melancholy menace of their tone!
        For every sound that floats
        From the rust within their throats
                 Is a groan.
        And the people—ah, the people—
       They that dwell up in the steeple,
                 All alone,
        And who tolling, tolling, tolling,
          In that muffled monotone,
         Feel a glory in so rolling
          On the human heart a stone—
     They are neither man nor woman—
     They are neither brute nor human—
              They are Ghouls:
        And their king it is who tolls;
        And he rolls, rolls, rolls,
                    Rolls
             A pæan from the bells!
          And his merry bosom swells
             With the pæan of the bells!
          And he dances, and he yells;
          Keeping time, time, time,
          In a sort of Runic rhyme,
             To the pæan of the bells—
               Of the bells:
          Keeping time, time, time,
          In a sort of Runic rhyme,
            To the throbbing of the bells—
          Of the bells, bells, bells—
            To the sobbing of the bells;
          Keeping time, time, time,
            As he knells, knells, knells,
          In a happy Runic rhyme,
            To the rolling of the bells—
          Of the bells, bells, bells—
            To the tolling of the bells,
      Of the bells, bells, bells, bells—
              Bells, bells, bells—
  To the moaning and the groaning of the bells.
\end{stanza}
\end{poem}

\pagebreak

\poemtitle{(Spring 2) \textbf{The Lover: A Ballad} XX}

\pagebreak

\poemtitle{(Spring 3) \textbf{The Sun Rising} XX}

\pagebreak

\poemtitle{(Spring 4) \textbf{Heart and Service} XX}

\pagebreak

\poemtitle{(Spring 5) \textbf{if everything happens that can't be done} X}
\attribution{e e cummings}
\begin{poem}
	\begin{stanza}
		if everything happens that can't be done\verseline
		(and anything's righter\verseline
		than books\verseline
		could plan)\verseline
		the stupidest teacher will almost guess\verseline
		(with a run\verseline
		skip\verseline
		around we go yes)\verseline
		there's nothing as something as one
	\end{stanza}
	\begin{stanza}
		one hasn't a why or because or although\verseline
		(and buds know better\verseline
		than books\verseline
		don't grow)\verseline
		one's anything old being everything new\verseline
		(with a what\verseline
		which\verseline
		around we come who)\verseline
		one's everyanything so
	\end{stanza}
	\begin{stanza}
		so world is a leaf so a tree is a bough\verseline
		(and birds sing sweeter\verseline
		than books\verseline
		tell how)\verseline
		so here is away and so your is a my\verseline
		(with a down\verseline
		up\verseline
		around again fly)\verseline
		forever was never till now
	\end{stanza}
	\pagebreak
	\begin{stanza}
		now i love you and you love me\verseline
		(and books are shuter\verseline
		than books\verseline
		can be)\verseline
		and deep in the high that does nothing but fall\verseline
		(with a shout\verseline
		each\verseline
		around we go all)\verseline
		there's somebody calling who's we
	\end{stanza}
	\begin{stanza}
		we're anything brighter than even the sun\verseline
		(we're everything greater\verseline
		than books\verseline
		might mean)\verseline
		we're everyanything more than believe\verseline
		(with a spin\verseline
		leap\verseline
		alive we're alive)\verseline
		we're wonderful one times one
	\end{stanza}
\end{poem}

\pagebreak

\poemtitle{(Spring 6) \textbf{\textit{I Met a Fool}} XX}

\pagebreak

\begin{volumetitlepage}
\sequencetitle{\textbf{II.} Summer}
\bigskip
\noindent\fbox{
	\parbox{\textwidth}{
		\poemtitle{(Cantrip) \textbf{A Drinking Song}}
		\attribution{William Butler Yeats}
		\begin{poem}
			\begin{stanza}
				Wine comes in at the mouth\\
				And love comes in at the eye;\\
				That’s all we shall know for truth\\
				Before we grow old and die.\\
				I lift the glass to my mouth,\\
				I look at you, and I sigh.
			\end{stanza}
		\end{poem}
	}
}
\end{volumetitlepage}

\pagebreak

\poemtitle{(Summer 1) \textbf{\textit{All Delights Are Vain}} XX}

\pagebreak

\poemtitle{(Summer 2) \textbf{When I Heard the Learn'd Astronomer} XX}

\pagebreak

\poemtitle{(Summer 3) \textbf{Love's Growth} XX}

\pagebreak

\poemtitle{(Summer 4) \textbf{The Time I've Lost In Wooing} XX}

\pagebreak

\poemtitle{(Summer 5) \textbf{Forget Not Yet The Tried Intent} XX}

\pagebreak

\poemtitle{(Summer 6) \textbf{I Wandered Lonely as a Cloud} XX}

\pagebreak

\begin{volumetitlepage}
\sequencetitle{\textbf{III.} Autumn}
\bigskip
\noindent\fbox{
	\parbox{\textwidth}{
		\poemtitle{(Cantrip) \textbf{The Road Not Taken}}
	}
}

\end{volumetitlepage}

\pagebreak

\poemtitle{(Autumn 1) \textbf{\textit{All The World's a Stage}} XX}
\attribution{Jaques, As You Like It (Shakespeare, }

\pagebreak

\poemtitle{(Autumn 2) \textbf{It Is Later Than You Think}}

\pagebreak

\poemtitle{(Autumn 3) \textbf{A Lecture Upon the Shadow} XX}

\pagebreak

\poemtitle{(Autumn 4) \textbf{Sailing to Byzantium} XX}

\pagebreak

\poemtitle{(Autumn 5) \textbf{A Dream Within a Dream} XX}

\pagebreak

\poemtitle{(Autumn 6) \textbf{Byzantium} XX}

\pagebreak

\begin{volumetitlepage}
\sequencetitle{\textbf{IV.} Winter}
\bigskip
\noindent\fbox{
	\parbox{\textwidth}{
		\poemtitle{(Cantrip) \textbf{Ozymandias} XX}
	}
}

\end{volumetitlepage}

\pagebreak

\poemtitle{(Winter 1) \textbf{Lines Written in Early Spring} XX}

\pagebreak

\poemtitle{(Winter 2) \textbf{Time to Come} XX}

\pagebreak

\poemtitle{(Winter 3) \textbf{I am A Little World Cunningly Made} XX}

\pagebreak

\poemtitle{(Winter 4) \textbf{The Conqueror Worm} XX}

\pagebreak

\poemtitle{(Winter 5) \textbf{Do Not Go Gentle} XX}

\pagebreak

\poemtitle{(Winter 6) \textbf{\textit{To Be or Not To Be}} XX}

\begin{volumetitlepage}
\sequencetitle{\textbf{V.} A Season Out Of Time}
\bigskip
\noindent\fbox{
	\parbox{\textwidth}{
		\poemtitle{(Cantrip) \textbf{Romance}}
	}
}

\end{volumetitlepage} 

\pagebreak

\poemtitle{(Epilogue 1) \textbf{The Ballad of God-Makers} XX}

\pagebreak

\poemtitle{(Epilogue 2) \textbf{The World is Too Much With Us} XX}
 
\pagebreak

\poemtitle{(Epilogue 3) \textbf{Love's Deity} XX}

\pagebreak

\poemtitle{(Epilogue 4) \textbf{Stopping by Woods on a Snowy Night} XX}

\pagebreak

\poemtitle{(Epilogue 5) \textbf{The Lovesong of J. Alfred Prufrock} XX}

\pagebreak

\poemtitle{(Epilogue 6) \textbf{Desiderata} XX}

\pagebreak

\end{document}